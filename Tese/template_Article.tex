\documentclass[]{article}
\usepackage[brazil]{babel}
\usepackage[utf8]{inputenc}
\usepackage{indentfirst}
%opening
\title{Análise Comparativa dos Algoritmos Genéticos Many-Objective em problemas de otimização discreta}
\author{Gabriel Augusto Marson\\Gina Maira Barbosa de Oliveira }

\begin{document}

\maketitle

\begin{abstract}
	
	\noindent Essa dissertação tem como objetivo mostrar o desempenho de alguns algoritmos genéticos em problemas clássicos da computação assim como em que circunstãncias são mais utilizados.
	
\end{abstract}

\section{Introdução}
	
	Publicada em 1859, a obra \textbf{Na Origem das Espécies – Sob o Conhecimento da Seleção Natural}, de Charles Darwin, foi alvo de críticas pois explicitava que o seres de um ambiente mudam de uma determinada geração pra outra e são selecionados de acordo com essas mudanças. A partir disso, Darwin escreveu que todos os seres vivos possuem um ancestral comum, contrastando com a visão criacionista da época.
	
	Essa teoria foi imprescindível para a criação dos algoritmos genéticos por John Henry Holland. Físico formado pelo MIT(Massachusetts Institute of Technology), Holland elaborou um algorítmo não determinístico com foco em resolução de problemas de custo muito alto.
	
	O AG consiste em dar uma interpretação para um conjunto de dados e evoluí-los para uma solução. As fases do AG dividem-se em aplicar uma função objetivo, a qual será responsável por classificar a aptidão do indivíduo, selecionar os individuos mais aptos, que serão os pais da próxima geração e realizar o cruzamento deles. Percebe-se a semelhança com a teoria de Darwin.
	
	Repete-se esse processo até que uma boa solução seja encontrada. Normalmente isso ocorre quando os indivíduos convergem para um só ou após um determinado número de gerações preestabelecido.
	
	Apesar de ter sido constatado o sucesso do AG para certos tipos de problemas, alguns outros permaneceram sem serem resolvidos pelo por esse método.
	
	

\end{document}
