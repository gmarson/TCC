\documentclass[]{article}
\usepackage[brazil]{babel}
\usepackage[utf8]{inputenc}
\usepackage{indentfirst}
%opening
\title{Análise Comparativa dos Algoritmos Genéticos Many-Objective em problemas de otimização discreta}
\author{Gabriel Augusto Marson\\ gabrielmarson@live.com }

\begin{document}

\maketitle

\begin{abstract}
	
	\noindent Essa dissertação tem como objetivo mostrar o desempenho de alguns algoritmos genéticos em problemas discretos clássicos da computação assim como em quais circunstâncias são mais utilizados.
	
\end{abstract}

\section{Introdução}

\subsection{Visão Geral}	

	Algoritmos genéticos são uma abordagem baseada na teoria da evolução de Darwin que têm como regra uma sequência de passos para atingir determinado objetivo.
	
	Apesar de ter sido constatado o sucesso do AG para certos tipos de problemas, alguns outros permaneceram sem serem resolvidos por esse método. Problemas de otimização discreta são um exemplo disso pois eles requerem que a classificação dos indivíduos seja realizada de acordo com vários critérios.
	
	Os AGs Multi-Objective constituem uma abordagem para resolver esse problema categorizando possíveis melhores respostas em virtude dos critérios avaliados.
	
	
\section{Agoritmos Genéticos}

	Publicada em 1859, a obra \textbf{Na Origem das Espécies – Sob o Conhecimento da Seleção Natural}, de Charles Darwin, foi alvo de críticas pois explicitava que o seres de um ambiente mudam de uma determinada geração pra outra e são selecionados de acordo com essas mudanças. A partir disso, Darwin escreveu que todos os seres vivos possuem um ancestral comum, contrastando com a visão criacionista da época.
	
	Essa teoria foi imprescindível para a criação dos algoritmos genéticos por John Henry Holland. Físico formado pelo MIT(Massachusetts Institute of Technology), Holland elaborou um algoritmo não determinístico com foco em resolução de problemas de custo muito alto.
	
	O AG consiste em dar uma interpretação para um conjunto de dados e evoluí-los para uma solução. As fases do AG dividem-se em aplicar uma função objetivo, a qual será responsável por classificar a aptidão do indivíduo, selecionar os indivíduos mais aptos, que serão os pais da próxima geração e realizar o cruzamento deles. Percebe-se a semelhança com a teoria de Darwin.
	
	Repete-se esse processo até que uma boa solução seja encontrada. Normalmente isso ocorre quando os indivíduos convergem para um só ou após um determinado número de gerações preestabelecido.

\end{document}
